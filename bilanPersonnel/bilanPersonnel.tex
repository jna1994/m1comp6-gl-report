\documentclass[a4paper,12pt]{article}
\usepackage[utf8x]{inputenc}
\usepackage[T1]{fontenc}
\usepackage[french]{babel}
\usepackage{graphicx}
\graphicspath{ {images/} }
\usepackage{float}
\usepackage{listings}
\usepackage{color}
\usepackage{url}

\setlength{\parindent}{0cm}
\setlength{\parskip}{1ex plus 0.5ex minus 0.2ex}

\title{Bilans personnels}

\begin{document}
\maketitle

Ce chapitre présente le bilan de chaque membre de l'équipe sur son expérience de l'agilité.



\section{Virgil}

Pour moi le projet Compilation/Génie Logiciel n'était pas une première fois car étant redoublant je l'ai aussi réalisé l'année dernière. J'ai voulu faire profiter de mon expérience à mes camarades de groupe dans un premier temps pour ensuite permettre au groupe de s'organiser correctement durant le projet.\\
Ce ne fut pas aussi simple que je ne l'avais imaginé. En faisant profiter de mon expérience aux autres, j'ai involontairement fait apparaitre certains problèmes que nous avions eu avec mon précédent groupe de l'année dernière, notamment d'un point de vue découpage du projet et répartition du travail. Cela nous a tout de même permis de bien nous organiser et de bien débuter la conception du projet. J'ai beaucoup appris à ce niveau-là, notamment par le fait que le groupe était sensiblement différent de celui dans lequel j'étais l'année dernière, ce qui impliquait donc que les personnes ne réagissaient pas de la même manière.\\
D'un point de vue plus technique, j'ai pu travailler sur d'autres tâches que celles que j'avais réalisé lors du projet de l'année dernière et ainsi avoir une meilleure compréhension du projet dans sa globalité.\\

\section{Mohammad}

Le travail a bien avancé même si parfois on a reporté des tâches au sprint suivant. Le découvert des outils (Maven, Jenkins, Nexus, SonarQube) est intéressant et nous apporte beaucoup de choses. C'est la première fois que je travaille avec une équipe avec plus de 3 personnes, et cela m'a permis d'apprendre beaucoup de choses en termes de connaissances techniques et de communication entre membre de l'équipe. En conclusion, le projet nous permet de découvrir le monde du développement de logiciel dans une équipe et je suis sur que cela nous prépare pour notre futur emploi dans l'informatique.

\section{Jérémy}
Les taches ont bien été découpées (chaque partie est indépendante et donc testable sans nécessairement passer par le GUI) mais, l'estimation de leur durée étant difficile, nous avons souvent compté large : la durée de réalisation de certaines taches ont été soit sous-estimées, soit sur-estimé.\\
La répartition des taches issues du découpage du projet a été selon moi relativement équilibré - du moins en théorie. En pratique, certains membres ont eu plus de travail que d'autres. Après, cette différence s'est surtout vue à des instants données. En ce qui le concerne, la partie GUI est relativement conséquente, mais la charge de travail était la plus élevée au début et à la fin du projet ("trou" à mi-projet).\\
Les outils de communication utilisés (Slack et Trello) m'ont paru utiles et pratiques. C'est dommage qu'on ne l'ait pas plus utilisé (moi le premier) au début.\\
Les réunions étaient fréquentes (minimum 1/semaine et la plupart du temps en dehors des créneaux de TP) et utiles. Seule contrariété : il a été difficile de réunir tout le monde à chaque fois. Les réunions étaient animées par Guzal, très bonnes contributions de Mohammad et Virgil - soit la moitié du groupe actif. Idem pour la prise de décision. Pour ma part, m'occupant que du GUI, je n'avais rien à ajouter. Donc je n'ai pas ressenti le besoin de m'exprimer plus que cela.\\
Enfin, l'arrivée de Medhi au milieu du projet est un peu difficile à gérer, surtout pour lui.\\
\paragraph{Suivi}
Il y a une grande disparité entre les sprints au niveau du rapport prévu / réalisé. Nous avons eu un peu de mal à débuter le projet (pas grand-chose de concret pour les sprints 1 et 2). La fin du projet est également tendue (période d'examen et de rendu de projets). Par contre, nous avons tenu les objectifs du premier release : nous nous en sommes très bien sortis lors de la démonstration en présence de Fabrice Bouquet.
\paragraph{Outils}
Départ difficile avec SVN (bug, problème d'accès).
\paragraph{Rétrospectives}
La rétrospective du premier release a été constructive (mise au point utilisation outils de communication, résolution des conflits de personne).


\section{Guzal}

Le choix d'un processus de développement itératif et incrémental est pertinent car il nous a permis de livrer une version intermédiaire fonctionnelle, avec retour du client. Cela permet un suivi des exigences fonctionnelles et c'est une source de motivation pour l'équipe. J'aurais donc préféré davantage de jalons intermédiaires.
Le découpage en sprints et en tâches améliore la collaboration et la livraison fréquente de valeur. En revanche, notre faible expérience en estimation du coût d'une tâche nous a obligé à reporter régulièrement des fonctionnalités d'un sprint au suivant.\\
Les rétrospectives et réunions hebdomadaires nous ont permis de mieux nous connaitre pour mieux collaborer et partager des opinions et des choix d'implémentation. J'aurais aimé plus de participation de chacun lors de ces moments.
Mon plus grand enseignement est d'ailleurs la difficulté à construire une équipe efficiente et auto-organisée. Pour cela il faut des gens motivés et responsables au sein d'une équipe stable. Nous avons eu à géré un départ et une arrivée et j'ai eu le sentiment que les motivations individuelles n'étaient pas à la réussite du projet. Mais je suis heureuse de cette expérience collaborative et je suis sûre qu'elle me sera utile pour évoluer dans ma prochaine équipe agile.

\end{document}