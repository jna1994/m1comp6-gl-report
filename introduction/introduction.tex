\documentclass[a4paper,12pt]{article}
\usepackage[utf8x]{inputenc}
\usepackage[T1]{fontenc}
\usepackage[french]{babel}
\usepackage{graphicx}
\graphicspath{ {images/} }
\usepackage{float}
\usepackage{listings}
\usepackage{color}
\usepackage{url}

\setlength{\parindent}{0cm}
\setlength{\parskip}{1ex plus 0.5ex minus 0.2ex}

\title{Rapport de conduite de projet\linebreak Introduction}

\begin{document}
\maketitle

Ce document présente des éléments de suivi du projet de compilation du groupe 6, dans le cadre du premier semestre de Master 1 en Informatique. 

Ce projet a pour objectif de réaliser en équipe un compilateur/interpréteur de MiniJaja au travers d'une interface utilisateur. Un premier document présente les choix techniques réalisés au cours de ce projet. Nous nous focaliserons ici sur les aspects de suivi du projet. Quels choix d'organisation ? Quelles méthodes de travail ont-elles été mises en œuvre ? Pour quels résultats et avec quels enseignements ?

Ce projet est une occasion pour nous d'appréhender des principes issues des méthodes agiles. En particuliers, ce projet a été la collaboration entre trois groupes d'acteurs qu'il nous semble important de présenter :
\begin{itemize}
\item le \emph{client} : représenté par Fabrice Bouquet, il est le destinataire du produit que nous livrons et le fournisseur des exigences fonctionnelles,
\item le \emph{product owner} : représenté par Elodie Bernard et Aymeric Cretin, pour suivre le bon avancement du projet,
\item l'équipe de développement : Adrien Begue, Awa Diallo, Guzal Khusanova, Virgil Manrique, Mohammad Nauval, Jérémy Navion et Mehdi Zemmoura.
\end{itemize} 

\medbreak

Dans ce rapport, nous évoquerons donc l'organisation, puis le suivi de ce projet et notamment des outils utilisés pour cela. Enfin, nous donnerons un bilan de cette première expérience de travail collaboratif en évoquant objectivement les réussites et les échecs rencontrés et les pistes éventuelles d'amélioration. 


\end{document}